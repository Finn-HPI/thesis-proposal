\section{Goal of Thesis}
\label{sec:goal}

% Description of the concrete problems addressed as the goals of the thesis, planned/expected results, reference to other thesis, paper, etc., if any.

This thesis aims to efficiently implement the sort-merge join algorithm,
explicitly optimized for specific architectures and hardware components. 
As Equi-joins are the most common type of join operation,
we will restrict ourselves to an Equi-join implementation and then optionally extend upon this later.

While multiple papers exist about modern implementation approaches for sort-merge joins in in-memory database systems
and SIMD sorting, only some have public 
implementations\footnote{Implementation of \cite{Balkesen} published at \url{https://archive-systems.ethz.ch/node/334}}.
Most SIMD sorting algorithms
presented in the literature are not directly applicable to join operations as they usually use
sorting keys of only 32 bits. We must additionally track the row ID (rid) corresponding to the sorting key for
a join, requiring at least 64-bit elements. The current implementations of sort-merge join in literature use SSE and AVX2 intrinsics,
but to our knowledge, there has yet to be an implementation using AVX-512. 

Therefore, in the scope
of this thesis, we want to integrate support for modern AVX-512 sorting algorithms (\cite{Watkins}, \cite{8855628})
next to SSE and AVX2 into a complete sort-merge join operator.
It would also be of value to see how new and existing approaches transfer to other CPU
architectures like Arm with its Scalable Vector Extension (SVE) or Power with its Vector Scalar
Extension (VSX). We are most interested in AWS Graviton 3/4 and Power9/10.

While some public implementations
exist for modern and optimized sort-merge join, they have usually isolated implementations with a strong
focus on the sorting step using randomly chosen input data, often already in the required data format. 
Also, they often skip the lookup of matching rows and the construction of the joined table.
Hence, in this thesis, we want to integrate our implementation of the sort-merge join into Hyrise
\cite{DBLP:conf/edbt/DreselerK0KUP19},
a research in-memory database. Hyrise contains both a radix-based hash-join and sort-merge join.
The sort-merge join uses radix cluster sorting, which uses pattern-defeating quicksort (boost)
but no explicit SIMD instructions. It fundamentally differs from the modern approaches in the literature.
These differences allow us to test our implementation against the existing sort-merge and hash-based join.
Complete integration into an in-memory database allows us to run decision support benchmarks
like TCP-H, TCP-DS, and the Public BI benchmark\footnote{\url{https://github.com/cwida/public_bi_benchmark}} to compare operators to other implementations in a 
more realistic scenario.

Benchmarks like TCP-H have schemas and datatypes carefully designed by experts in database design.
Hence, they can fail to capture the chaotic nature of real-world applications \cite{10.1145/3209950.3209952}.
For instance, TCP-H only uses integer values for keys. 32-bit integer values do not require any change
in data format to be SIMD sortable. However, in many Business Intelligence applications, strings
are used for various types, e.g., to deal with dirty data that is not parsable. String join keys
complicate SIMD sorting, as multiple strings do not fit into SIMD registers. Therefore, we must
reduce the key size by compression, prefix functions, or hashing. Shortening the key size can
introduce false positives, which need to be filtered. We could accelerate traditional string
sorting through SIMD in other ways (e.g., SIMD accelerated string comparison), not requiring any
form of compression, but methods like sorting networks and bitonic merge networks will likely not
be applicable.

As strings are variable in size, it makes sense to consider their internal representation and
encoding. For instance, in a dictionary encoding, the indices can be used as a sorting key rather
than the string itself. Other representations allow for cheap access to a prefix or short string.
For instance, Umbra's ``German String'' \cite{DBLP:conf/cidr/NeumannF20} consists of a 128-bit 
struct and are adopted by more recent
databases like CedarDB and DuckDB. The first 32 bits represent the length. The remaining bits hold
the complete string if the length is at most 12. Otherwise, the struct consists of the 32-bit
length, a 32-bit prefix, and a pointer to a storage location. Due to saving pointer dereferences,
this can speed up comparison, lexicographical sorting, and other prefix operations.

Benchmarking should also include measuring the sorting throughput in tuples per second and all 
algorithmic steps: initial data construction in the format of (key, rid) from the input relations,
sorting, finding join partners, and the final construction of the joined table. 
We want to test our implementation on different architectures and hardware, evaluating
differences in core count, cache size, SIMD registers widths, NUMA regions, 
and other hardware-specific properties.
