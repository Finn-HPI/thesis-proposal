\section{Goal of Thesis}
\label{sec:goal}

% Description of the concrete problems addressed as the goals of the thesis, planned/expected results, reference to other thesis, paper, etc., if any.
This thesis aims to efficiently implement the sort-merge join algorithm, explicitly optimized 
for specific architectures and hardware components. As equi-joins are the most common type of 
join operation \cite{DBLP:conf/cloud/BlanasP13}, we will restrict ourselves to an equi-join implementation and then optionally
extend upon this. We will examine the join in isolation without considering the implications of the sorted results.

The goals of this thesis include an open-source implementation of a sort-merge join algorithm
integrated into the Hyrise in-memory database. The sorting should be accelerated through SIMD
parallelism and support at least AVX2 and AVX-512. Possible input data types are int, float, and string
data. We also want to experiment with other architectures like Arm and Power and run benchmarks to
evaluate our approach and compare it to other join implementations
(e.g., existing Hyrise sort-merge join, and other state-of-the-art sort-merge join).

\subsection{Hyrise Integration}
While some public implementations
exist for modern and optimized sort-merge joins, they have usually isolated implementations with a strong
focus on the sorting step using randomly chosen input data, often already in the required data format. 
Also, they often skip the lookup of matching rows and the construction of the joined table.
Hence, in this thesis, we want to integrate our implementation of the sort-merge join into Hyrise
\cite{DBLP:conf/edbt/DreselerK0KUP19},
a research in-memory database. Hyrise contains both a radix-based hash-join and sort-merge join.
The sort-merge join uses radix cluster sorting, which uses ``pdqsort'' (\href{https://www.boost.org/}{Boost C++ library})
but no explicit SIMD instructions. It fundamentally differs from the modern approaches in the literature.
These differences allow us to test our implementation against the existing sort-merge and hash-based join.

\subsection{SIMD Sorting}

Most SIMD sorting algorithms presented in the literature use sorting keys of only 32 bits. We must additionally track the row ID
(rid) corresponding to the sorting key for a join, resulting in 64-bit elements. Most architectures
only support 8-, 16-, 32-, and 64-bit elements for SIMD. Therefore, larger elements like 128-bit are
usually not supported.

The current implementations of sort-merge join in literature use SSE and AVX2 intrinsics, but to
our knowledge, there has yet to be an implementation using AVX-512. Therefore, in the scope of
this thesis, we want to integrate support for modern AVX-512 sorting algorithms \cite{Watkins, 8855628}. 

\subsection{Architectures}
It would also be of value to see how new and existing approaches transfer to other CPU
architectures like Arm with its Scalable Vector Extension (SVE) or Power with its Vector Scalar
Extension (VSX).

Hence, in addition to x86 systems, we optinally want to experiment with AWS Graviton 4 or other ARM chips
and Power10. This requires adapting 
the implementation to use the respective architecture's
vector extension.


\subsection{Benchmarking}
Complete integration into an in-memory database allows us to run decision support benchmarks
like TCP-H, TCP-DS, and the Join Order Benchmark (JOB) \cite{DBLP:journals/pvldb/LeisGMBK015} to
compare operators to other implementations in a more realistic scenario.

Benchmarks like TCP-H have schemas and datatypes carefully designed by experts in database design.
Hence, they can fail to capture the chaotic nature of real-world applications \cite{10.1145/3209950.3209952}.
For instance, TCP-H only uses integer values for keys. 
% 32-bit integer values do not require any change
% in data format to be SIMD sortable.

However, many business intelligence applications use strings for various data types, e.g., to deal
with dirty data that is not parsable. Hence, we frequently encounter string-type join keys. For
that, we chose JOB due to its real-world data and ease of use, as it 
has already been integrated into Hyrise. The Public BI benchmark\footnote{\url{https://github.com/cwida/public_bi_benchmark}}
also provides a realistic setting
but is more difficult to use, and we only consider it optional.

% String join keys
% complicate SIMD sorting, as multiple strings do not fit into SIMD registers. Therefore, we must
% reduce the key size by compression, prefix functions, or hashing. Shortening the key size can
% introduce false positives, which need to be filtered. We could accelerate traditional string
% sorting through SIMD in other ways (e.g., SIMD accelerated string comparison), not requiring any
% form of compression, but methods like sorting networks and bitonic merge networks will likely not
% be applicable.

Benchmarking should also include measuring the sorting throughput in tuples per second and all 
algorithmic steps: initial data construction in the format of (key, rid) from the input relations,
sorting, finding join partners, and the final construction of the joined table. 


