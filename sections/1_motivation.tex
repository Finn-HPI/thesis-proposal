\section{Motivation}
\label{sec:motivation}

Historically, many systems were designed for computer systems and architectures where I/O dominates
performance. However, modern processors with multi-core architectures, advanced 
instruction sets, and other hardware accelerants like vector operations, which allow us to
simultaneously perform the same operation on multiple data items, have significantly altered this
landscape. Hence, we must reconsider how we implement modern systems.

\subsection{Importance for Database Systems}

Many (in-memory) database systems are no longer I/O bound and, therefore, need high intra-operator
parallelism. To fully utilize the multi-core architecture and other hardware features such as cache
locality and SIMD instructions for higher data parallelism should be considered to achieve maximum
performance.

\subsection{Efficient Join Implementation}
The join operator is a fundamental component of every database system.
In recent years, the difference in performance between the sort-merge and radix-hash join has been
the subject of ongoing debate. Kim et al. \cite{10.14778/1687553.1687564} projected that Sort-Merge Join would outperform hash-based
alternatives with a factor of $1.35$ – $1.65$ with 512-bit SIMD. Albutiu et al. \cite{MPSM} reinforced this claim with recent results reporting that
their NUMA-aware implementation of sort-merge join is superior to that of hash joins (without
leveraging SIMD). Balkesen et al. \cite{Balkesen} experimentally show contradicting results by implementing 
optimized versions for sort-merge and radix-hash join, showing that their implementation of
radix-hash join is still superior. They use AVX2 in their implementation, allowing further
work to explore wider SIMD registers (e.g., AVX-512).

\subsection{Open-Source Implementation}

Despite ongoing research, public implementations of join algorithms optimized for modern hardware
are hard to find. Most existing implementations are proprietary or 
experimental\footnote{Implementation by Balkesen et al.: \url{https://archive-systems.ethz.ch/node/334}}, limiting their
accessibility and usefulness to the research community and database developers. An open-source,
state-of-the-art implementation of a sort-merge join optimized for different architectures would
help address this need. Such an implementation serves as a valuable baseline for researchers
looking to evaluate or improve upon existing methods, and it also contributes to advancing database
system design by providing a solid foundation for future innovation.


% Description of the scientific context (including the classification in the literature, projects, ...) and of the concrete embedding.

